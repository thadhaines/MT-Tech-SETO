\documentclass[12pt]{article}
\usepackage[english]{babel}
\usepackage[utf8]{inputenc}

%% Pointer to 'default' preamble
% pacakages and definitions

\usepackage{geometry}
\geometry{
	letterpaper, 
	portrait, 
	top=.75in,
	left=.8in,
	right=.75in,
	bottom=.5in		} 	% Page Margins
	
%% additional packages for nice things
\usepackage{amsmath} 	% for most math
\usepackage{commath} 	% for abs
\usepackage{lastpage}	% for page count
\usepackage{amssymb} 	% for therefore
\usepackage{graphicx} 	% for image handling
\usepackage{wrapfig} 	% wrap figures
\usepackage[none]{hyphenat} % for no hyphenations
\usepackage{array} 		% for >{} column characterisctis
\usepackage{physics} 	% for easier derivative \dv....
\usepackage{tikz} 		% for graphic@!
\usepackage{circuitikz} % for circuits!
\usetikzlibrary{arrows.meta} % for loads
\usepackage[thicklines]{cancel}	% for cancels
\usepackage{xcolor}		% for color cancels
\usepackage[per-mode=fraction]{siunitx} % for si units and num
\sisetup{group-separator = {,}, group-minimum-digits = 3} % additional si unit table functionality

\usepackage{fancyhdr} 	% for header
\usepackage{comment}	% for ability to comment out large sections
\usepackage{multicol}	% for multiple columns using multicols
\usepackage[framed,numbered]{matlab-prettifier} % matlab sytle listing
\usepackage{marvosym} 	% for boltsymbol lightning
\usepackage{pdflscape} 	% for various landscape pages in portrait docs.
%\usepackage{float}
\usepackage{fancyvrb}	% for Verbatim (a tab respecting verbatim)
\usepackage{enumitem}	% for [resume] functionality of enumerate
\usepackage{spreadtab} 	% for using formulas in tables}
\usepackage{numprint}	% for number format in spread tab
\usepackage{subcaption} % for subfigures with captions
\usepackage[normalem]{ulem} % for strike through sout

% for row colors in tables....
\usepackage{color, colortbl}
\definecolor{G1}{gray}{0.9}
\definecolor{G2}{rgb}{1,0.88,1}%{gray}{0.6}
\definecolor{G3}{rgb}{0.88,1,1}

% For table formatting
\usepackage{booktabs}
\renewcommand{\arraystretch}{1.2}
\usepackage{floatrow}
\floatsetup[table]{capposition=top} % put table captions on top of tables

% Caption formating footnotesize ~ 10 pt in a 12 pt document
\usepackage[font={small}]{caption}

%% package config 
\sisetup{output-exponent-marker=\ensuremath{\mathrm{E}}} % for engineer E
\renewcommand{\CancelColor}{\color{red}}	% for color cancels
\lstset{aboveskip=2pt,belowskip=2pt} % for more compact table
%\arraycolsep=1.4pt\def
\setlength{\parindent}{0cm} % Remove indentation from paragraphs
\setlength{\columnsep}{0.5cm}
\lstset{
	style      = Matlab-editor,
	basicstyle = \ttfamily\footnotesize, % if you want to use Courier - not really used?
}
\renewcommand*{\pd}[3][]{\ensuremath{\dfrac{\partial^{#1} #2}{\partial #3}}} % for larger pd fracs
\renewcommand{\real}[1]{\mathbb{R}\left\{ #1 \right\}}	% for REAL symbol
\newcommand{\imag}[1]{\mathbb{I}\left\{ #1 \right\}}	% for IMAG symbol
\definecolor{m}{rgb}{1,0,1}	% for MATLAB matching magenta
	
%% custom macros
\newcommand\numberthis{\addtocounter{equation}{1}\tag{\theequation}} % for simple \numberthis command

\newcommand{\equal}{=} % so circuitikz can have an = in the labels
\newcolumntype{L}[1]{>{\raggedright\let\newline\\\arraybackslash\hspace{0pt}}m{#1}}
\newcolumntype{C}[1]{>{\centering\let\newline\\\arraybackslash\hspace{0pt}}m{#1}}
\newcolumntype{R}[1]{>{\raggedleft\let\newline\\\arraybackslash\hspace{0pt}}m{#1}}

%% Header
\pagestyle{fancy} % for header stuffs
\fancyhf{}
% spacing
\headheight 29 pt
\headsep 6 pt

%% Header
\rhead{Thad Haines \\ Page \thepage\ of \pageref{LastPage}}
\chead{Variable Step Size PST Results \\ Compared to Fixed Step PST Results}
\lhead{Research \\ 7/28/20}

\usepackage{setspace}
\usepackage{multicol}
\begin{document}
\onehalfspacing
\paragraph{Scenario} \ \\
A 14 ms 3 Phase Fault in the New England 39 Bus, 10 Machine Benchmark System was simulated using the standard PST fixed time step method (Huen's Method) and five of the standard MATLAB ODE solvers that employ variable time step (VTS) methods.
The simulation was only 20 seconds to verify transient dynamics could be captured, and confirm if the VTS methods increased the time step in a desirable way.

\paragraph{Summary} 
\begin{itemize}
\item Variable time step (VTS) simulation works in PST.
\item Bus Voltage and Angle, Machine Speed, Average System Frequency, Exciter Efd, and PSS out dynamics from the VTS simulations seem to match fixed results well.
\item Some methods do not work (ODE45, ODE23s)
\item For this relatively short simulation (20 Seconds), all VTS simulations took $\approx$3 times longer than the standard PST fixed method.
\item The VTS methods rely on tolerances to change the time step (no minimum step size option).
\item Some steps require hundreds of network and dynamic solutions before being accepted.
\end{itemize}


\begin{table}[!ht]
\resizebox{\linewidth}{!}{
	\centering
	\begin{tabular}{@{} L{1.75cm} 
	R{2cm} R{2cm}  R{2cm} R{1.5cm} R{0.75cm} R{0.75cm} R{1.5cm} R{2cm} R{2cm}@{}} 	
		\toprule % @ signs to remove extra L R space
		\footnotesize % this will affect the table font (makse it 10pt)
		\raggedright % for non justified table text

	&	\multicolumn{3}{c}{Step Size [seconds]}					&		&	\multicolumn{2}{c}{\shortstack{Solutions\\ Per Step}}			&		&		&		\\	
Method	&	Max	&	Min	&	Ave	&	Total Steps	&	Ave.	&	Max.	&	Total Slns	&	Sim time	&	Speed Up	\\ \midrule	
Fixed	&	0.020	&	0.02	&	0.0200	&	1000	&	2	&	2	&	2000	&	8.1207	&	1	\\	
ODE113	&	0.133	&	5.16E-05	&	0.0148	&	1350	&	2	&	7	&	2937	&	19.849	&	0.409	\\	
ODE15s	&	0.110	&	1.37E-05	&	0.0276	&	725	&	8	&	382	&	5715	&	25.6757	&	0.316	\\	
ODE23	&	0.185	&	1.27E-05	&	0.0216	&	925	&	3	&	15	&	3023	&	17.6041	&	0.461	\\	
ODE23t	&	0.372	&	3.73E-05	&	0.0266	&	753	&	6	&	382	&	4817	&	23.178	&	0.350	\\	
ODE23tb	&	0.546	&	1.51E-05	&	0.0327	&	612	&	9	&	383	&	5645	&	26.278	&	0.309	\\	\bottomrule
	\end{tabular}
	}%end resize box
\end{table}

\paragraph{Observations of Note}
\begin{enumerate}
%\item   ODE45 not compatible with VTS routine(returns 4 solutions per integration step)
%\item ODE 23s requires ~ 200 solutions per step (not useful)
\item A smaller initial step 'may' reduce the number of solutions required at the beginning of 'time blocks'. Alternatively, tolerances may be adjust to further 'tune' simulation operation.
\item It should be possible to change the solution method and/or ODE solver during simulation.
\item A longer simulation will probably highlight the benefits of VTS better.

\end{enumerate}


\pagebreak
% fixed: 8.1207 seconds total, 1000 steps i.e., 2k network solutions (2 solutions per step)
%ode113 time: 19.849
%ode15s time: 25.6757
%ode23 time: 17.6041
%ode23t time: 23.178
%ode23tb time: 26.278
 
\foreach \name in {ode113, ode15s, ode23, ode23t, ode23tb}{
\subparagraph{\name \ Results} \ \\
\includegraphics[width=\linewidth]{\name comp} \\

\includegraphics[width=\linewidth]{\name steps} 
\pagebreak
}% end for each?

\end{document}
