\documentclass[12pt]{article}
\usepackage[english]{babel}
\usepackage[utf8]{inputenc}

%% Pointer to 'default' preamble
% pacakages and definitions

\usepackage{geometry}
\geometry{
	letterpaper, 
	portrait, 
	top=.75in,
	left=.8in,
	right=.75in,
	bottom=.5in		} 	% Page Margins
	
%% additional packages for nice things
\usepackage{amsmath} 	% for most math
\usepackage{commath} 	% for abs
\usepackage{lastpage}	% for page count
\usepackage{amssymb} 	% for therefore
\usepackage{graphicx} 	% for image handling
\usepackage{wrapfig} 	% wrap figures
\usepackage[none]{hyphenat} % for no hyphenations
\usepackage{array} 		% for >{} column characterisctis
\usepackage{physics} 	% for easier derivative \dv....
\usepackage{tikz} 		% for graphic@!
\usepackage{circuitikz} % for circuits!
\usetikzlibrary{arrows.meta} % for loads
\usepackage[thicklines]{cancel}	% for cancels
\usepackage{xcolor}		% for color cancels
\usepackage[per-mode=fraction]{siunitx} % for si units and num
\sisetup{group-separator = {,}, group-minimum-digits = 3} % additional si unit table functionality

\usepackage{fancyhdr} 	% for header
\usepackage{comment}	% for ability to comment out large sections
\usepackage{multicol}	% for multiple columns using multicols
\usepackage[framed,numbered]{matlab-prettifier} % matlab sytle listing
\usepackage{marvosym} 	% for boltsymbol lightning
\usepackage{pdflscape} 	% for various landscape pages in portrait docs.
%\usepackage{float}
\usepackage{fancyvrb}	% for Verbatim (a tab respecting verbatim)
\usepackage{enumitem}	% for [resume] functionality of enumerate
\usepackage{spreadtab} 	% for using formulas in tables}
\usepackage{numprint}	% for number format in spread tab
\usepackage{subcaption} % for subfigures with captions
\usepackage[normalem]{ulem} % for strike through sout

% for row colors in tables....
\usepackage{color, colortbl}
\definecolor{G1}{gray}{0.9}
\definecolor{G2}{rgb}{1,0.88,1}%{gray}{0.6}
\definecolor{G3}{rgb}{0.88,1,1}

% For table formatting
\usepackage{booktabs}
\renewcommand{\arraystretch}{1.2}
\usepackage{floatrow}
\floatsetup[table]{capposition=top} % put table captions on top of tables

% Caption formating footnotesize ~ 10 pt in a 12 pt document
\usepackage[font={small}]{caption}

%% package config 
\sisetup{output-exponent-marker=\ensuremath{\mathrm{E}}} % for engineer E
\renewcommand{\CancelColor}{\color{red}}	% for color cancels
\lstset{aboveskip=2pt,belowskip=2pt} % for more compact table
%\arraycolsep=1.4pt\def
\setlength{\parindent}{0cm} % Remove indentation from paragraphs
\setlength{\columnsep}{0.5cm}
\lstset{
	style      = Matlab-editor,
	basicstyle = \ttfamily\footnotesize, % if you want to use Courier - not really used?
}
\renewcommand*{\pd}[3][]{\ensuremath{\dfrac{\partial^{#1} #2}{\partial #3}}} % for larger pd fracs
\renewcommand{\real}[1]{\mathbb{R}\left\{ #1 \right\}}	% for REAL symbol
\newcommand{\imag}[1]{\mathbb{I}\left\{ #1 \right\}}	% for IMAG symbol
\definecolor{m}{rgb}{1,0,1}	% for MATLAB matching magenta
	
%% custom macros
\newcommand\numberthis{\addtocounter{equation}{1}\tag{\theequation}} % for simple \numberthis command

\newcommand{\equal}{=} % so circuitikz can have an = in the labels
\newcolumntype{L}[1]{>{\raggedright\let\newline\\\arraybackslash\hspace{0pt}}m{#1}}
\newcolumntype{C}[1]{>{\centering\let\newline\\\arraybackslash\hspace{0pt}}m{#1}}
\newcolumntype{R}[1]{>{\raggedleft\let\newline\\\arraybackslash\hspace{0pt}}m{#1}}

%% Header
\pagestyle{fancy} % for header stuffs
\fancyhf{}
% spacing
\headheight 29 pt
\headsep 6 pt

%% Header
\rhead{Thad Haines \\ Page \thepage\ of \pageref{LastPage}}
\chead{AGC Notes\\ }
\lhead{Research \\ 6/19/20}

\usepackage{minted}
\usepackage{setspace}
\begin{document}
\onehalfspacing
\paragraph{General AGC Notes} \ \\
The current versions of Power System Toolbox (PST) does not have any AGC (automatic generation control) or area related models. 
Previous AGC work from PSLTDSim (Power system long-term dynamic simulator) may be a good starting point for adding AGC to PST. \\

The central computational task behind AGC is the calculation and distribution of an area control error (ACE) value.
At the most basic level, this requires knowing an area’s actual net interchange, scheduled net interchange, and frequency.
The standard NERC reported ACE (RACE) calculation is 
\begin{equation}{\label{eq: race} }
RACE = (NI_A - NI_S) - 10B (F_A - F_S) - I_{ME} + I_{ATEC}  
\end{equation}%{Area Control Error} % this function creates a new paragraph.
With variables defined as:
\begin{itemize}
\item $NI_A$ - Actual net interchange
\item $NI_S$ - Scheduled net interchange
\item $B$ - Fixed frequency bias
\item $F_A$ - Actual system frequency
\item $F_S$ - Scheduled system frequency
\item $I_{ME}$ - Meter error 
\item $I_{ATEC}$ - Area time error correction
\end{itemize}
$I_{ME}$ and $I_{ATEC}$ are often ignored, or set to zero, for simulation purposes.

An optional variable frequency bias $B_V$  that replaces/enhances the standard frequency bias $B$ may be calculated as 
\begin{equation}{\label{eq:varibaleBias} }
B_V = B\left( 1 + K_B \abs{\Delta \omega} \right)
\end{equation}%\eqcaption{Variable Frequency Bias} % this function creates a new paragraph.
Where
\begin{equation}{\label{eq:speed-deviation} }
\Delta\omega = \omega_{rated} - \omega.
\end{equation}%\eqcaption{Per-Unit Speed Deviation}
It should be noted that frequency bias is a negative number and has units of $MW/0.1Hz$.

By convention, $ACE$, or $RACE$ is measured in $MW$ and a positive value indicates a general over-generation condition while a negative value is representative of an under-generation situation.

\pagebreak
\paragraph{PST possibilities} \ \\
To accommodate AGC, PST would require an ‘\verb|area_con|’ data structure that lists each bus associated with a particular area. 
Code will have to be written to calculate and log area interchange based on power flow through lines that connect areas. 
These calculations are already included with PST in the \verb|line_pq| function, but used only in the linearization of systems and after a non-linear simulation.\\

Some kind of center of inertia combined frequency (from Sam’s phd work) may be used to  estimate area or system frequency used in ACE calculations.
This would also require area generator buses to be defined in the \verb|area_con|.\\

An ‘\verb|agc_con|’ will also have to be created to define the parameters and options associated with AGC such as frequency bias, action time, filter parameters, integration parameters, gains, etc. 
A list of AGC controlled generators (with participation factors) would also be required. \\

%Variable frequency bias options may be included, though not required.

Instead of using the previous ‘array’ based definitions that PST has historically employed for system parameterization, a structured array approach may be used i.e.

\begin{minted}[
		frame=lines,
		framesep=2mm,
		baselinestretch=1.2,
		bgcolor=gray!13,
		fontsize=\footnotesize,
		%linenos,
		breaklines
		]{MATLAB}
agc_con(1).actTime = 30; % AGC action time for AGC 1
…
agc_con(2).ctrlGens = [2,3,4]; % list of AGC gens for AGC 2
agc_con(2).genPF = [0.25, 0.25, 0.5]; % participation factor for AGC2
…
\end{minted}
For reference, the AGC model in PSLTDSim is created using code similar to:


\begin{minted}[
		frame=lines,
		framesep=2mm,
		baselinestretch=1.2,
		bgcolor=gray!13,
		fontsize=\footnotesize,
		%linenos,
		breaklines
		]{python}
mirror.sysBA = {
    'BA1':{
        'Area':1,
        'B': "1.0 : permax", # MW/0.1 Hz
        'BVgain' : 500.0, # variable frequency bias gain
        'AGCActionTime': 30.00, # seconds 
        'ACEgain' : 2.0,
        'AGCType':'TLB : 0', # Tie-Line Bias 
        'IncludeIACE' : True,
        'IACEwindow' : 30, # seconds - size of window - 0 for non window
        'IACEscale' : 1/10,
        'ACEFiltering': 'PI : 0.04 0.0001'
        'CtrlGens': [
            'gen 1 : 0.5 : rampA',
            'gen 2 1 : 0.5 : rampA', ]
        },
    }
        \end{minted}     
        
\pagebreak
\paragraph{PSLTDSim AGC model} \ \\
The AGC model from PSLTDSim contains basic and optional control blocks. 
% insert fig here
\begin{figure}[!ht]
	\centering
	\footnotesize
	\includegraphics[width=\linewidth]{../../../../../ResearchDocs/TEX/models/AGC-TLB/AGC-TLB}
	\caption{Block diagram of ACE calculation and manipulation.}
	\label{fig: AGC-TLB}
\end{figure}%\vspace{-1em} % will remove 1 white space after image - typically good


The conditional summing block contains logic that may be used to collect error values that are deemed internal to an area to more efficiently respond to events.
This was done by comparing the sign of the error signal to the sign of frequency deviation.\\
 
A moving window integrator with a user defined time range was used in the PSLTDSim AGC model. 
This seems possible in PST, but will require some extra considerations when dealing with variable time steps.\\

In simulations carried out by PSLTDSim, the most useful optional filter was the proportional-integral (PI) type shown below.
% insert fig here
\begin{figure}[!ht]
	\centering
	\footnotesize
	\includegraphics[width=\linewidth]{../../../../../ResearchDocs/TEX/models/filterAgents/filterAgent}
	\caption{Block diagrams of optional ACE filters (Low pass, Integral, PI).}
	\label{fig: filterAgents}
\end{figure}%\vspace{-1em} % will remove 1 white space after image - typically good


ACE signals distributed to participating generators acted as a relative ramp on a the associated governor $P_{REF}$ value (or $P_{mech}$ for un-governed units).
A low pass filter with a large time constant $\approx$1/4  AGC action time may be an easier implementation for PST.\\

Plant controllers (similar to GDACS) were not created in PSLTDSim due to time constraints and may be overly detailed for this project.
\end{document}
