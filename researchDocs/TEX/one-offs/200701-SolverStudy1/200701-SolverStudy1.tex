\documentclass[12pt]{article}
\usepackage[english]{babel}
\usepackage[utf8]{inputenc}

%% Pointer to 'default' preamble
% pacakages and definitions

\usepackage{geometry}
\geometry{
	letterpaper, 
	portrait, 
	top=.75in,
	left=.8in,
	right=.75in,
	bottom=.5in		} 	% Page Margins
	
%% additional packages for nice things
\usepackage{amsmath} 	% for most math
\usepackage{commath} 	% for abs
\usepackage{lastpage}	% for page count
\usepackage{amssymb} 	% for therefore
\usepackage{graphicx} 	% for image handling
\usepackage{wrapfig} 	% wrap figures
\usepackage[none]{hyphenat} % for no hyphenations
\usepackage{array} 		% for >{} column characterisctis
\usepackage{physics} 	% for easier derivative \dv....
\usepackage{tikz} 		% for graphic@!
\usepackage{circuitikz} % for circuits!
\usetikzlibrary{arrows.meta} % for loads
\usepackage[thicklines]{cancel}	% for cancels
\usepackage{xcolor}		% for color cancels
\usepackage[per-mode=fraction]{siunitx} % for si units and num
\sisetup{group-separator = {,}, group-minimum-digits = 3} % additional si unit table functionality

\usepackage{fancyhdr} 	% for header
\usepackage{comment}	% for ability to comment out large sections
\usepackage{multicol}	% for multiple columns using multicols
\usepackage[framed,numbered]{matlab-prettifier} % matlab sytle listing
\usepackage{marvosym} 	% for boltsymbol lightning
\usepackage{pdflscape} 	% for various landscape pages in portrait docs.
%\usepackage{float}
\usepackage{fancyvrb}	% for Verbatim (a tab respecting verbatim)
\usepackage{enumitem}	% for [resume] functionality of enumerate
\usepackage{spreadtab} 	% for using formulas in tables}
\usepackage{numprint}	% for number format in spread tab
\usepackage{subcaption} % for subfigures with captions
\usepackage[normalem]{ulem} % for strike through sout

% for row colors in tables....
\usepackage{color, colortbl}
\definecolor{G1}{gray}{0.9}
\definecolor{G2}{rgb}{1,0.88,1}%{gray}{0.6}
\definecolor{G3}{rgb}{0.88,1,1}

% For table formatting
\usepackage{booktabs}
\renewcommand{\arraystretch}{1.2}
\usepackage{floatrow}
\floatsetup[table]{capposition=top} % put table captions on top of tables

% Caption formating footnotesize ~ 10 pt in a 12 pt document
\usepackage[font={small}]{caption}

%% package config 
\sisetup{output-exponent-marker=\ensuremath{\mathrm{E}}} % for engineer E
\renewcommand{\CancelColor}{\color{red}}	% for color cancels
\lstset{aboveskip=2pt,belowskip=2pt} % for more compact table
%\arraycolsep=1.4pt\def
\setlength{\parindent}{0cm} % Remove indentation from paragraphs
\setlength{\columnsep}{0.5cm}
\lstset{
	style      = Matlab-editor,
	basicstyle = \ttfamily\footnotesize, % if you want to use Courier - not really used?
}
\renewcommand*{\pd}[3][]{\ensuremath{\dfrac{\partial^{#1} #2}{\partial #3}}} % for larger pd fracs
\renewcommand{\real}[1]{\mathbb{R}\left\{ #1 \right\}}	% for REAL symbol
\newcommand{\imag}[1]{\mathbb{I}\left\{ #1 \right\}}	% for IMAG symbol
\definecolor{m}{rgb}{1,0,1}	% for MATLAB matching magenta
	
%% custom macros
\newcommand\numberthis{\addtocounter{equation}{1}\tag{\theequation}} % for simple \numberthis command

\newcommand{\equal}{=} % so circuitikz can have an = in the labels
\newcolumntype{L}[1]{>{\raggedright\let\newline\\\arraybackslash\hspace{0pt}}m{#1}}
\newcolumntype{C}[1]{>{\centering\let\newline\\\arraybackslash\hspace{0pt}}m{#1}}
\newcolumntype{R}[1]{>{\raggedleft\let\newline\\\arraybackslash\hspace{0pt}}m{#1}}

%% Header
\pagestyle{fancy} % for header stuffs
\fancyhf{}
% spacing
\headheight 29 pt
\headsep 6 pt

%% Header
\rhead{Thad Haines \\ Page \thepage\ of \pageref{LastPage}}
\chead{MATLAB variable step ode solver \\ Compared to fixed step lsim using a state space PSS model}
\lhead{Research \\ 7/01/20}

\usepackage{minted}
\usepackage{setspace}
\usepackage{lscape}
\usepackage{multicol}
\begin{document}
\onehalfspacing
\paragraph{Step of a 3rd Order State Space System} \ \\
The previous ODE45 comparison study of a step input to a PSS model was altered to test a variety of variable step MATLAB ode solvers.
Simulation time was set to one minute so that step time variations could be observed after a disturbance.
It should be noted that maximum step size was limited to 20 seconds (20,000 ms).

\paragraph{Summary} \ \\
The table below shows the number of steps each method took for a 60 second simulation, the maximum step size post-step event, the magnitude of the steady state error, and if the method is Octave compatible.
Full result plots are presented in the following pages with MATLAB code at the end of this document.

\begin{table}[!ht]
	\centering
	\begin{tabular}{@{} L{2cm} 
	R{2cm} R{4cm}  R{3cm} C{3cm}@{}} 	
		\toprule % @ signs to remove extra L R space
		\footnotesize % this will affect the table font (makse it 10pt)
		\raggedright % for non justified table text
Method & 	Number of Steps & 	 Max Step Post Disturbance [ms]	& SS Error Magnitude & Octave compatible\\ \midrule
Fixed & 14,401 & $\approx$4 & - & *\\
ODE45 & 2,426 & 30 & $10^{-4}$ & *\\
ODE23 & 803 & 75 & $10^{-4}$ & * \\
ODE113 & 1,298 & 75 & $10^{-4}$\\
ODE15s & 71 & 20,000 & $10^{-8}$ & *\\
ODE23s & 45 & 20,000 & $10^{-9}$\\
ODE23t & 72 & $\approx$17,000 & $10^{-9}$\\
ODE23tb & 49 & 20,000 & $10^{-8}$\\

		\bottomrule
	\end{tabular}
\end{table}

\paragraph{Observations of Note}
\begin{enumerate}
\item All ODE methods greatly reduce the number of required steps.
\item ODE15s, ODE23s, and ODE23tb reached the maximum allowed step size of 20 seconds.
\item Steady state error for ODE45, ODE23, and ODE113 was approximately 4 orders of magnitude larger than all other methods and step size stayed below 75 ms.
\item ODE23s used the least amount of steps and had one of the smallest steady state errors.
\item ODE15s appears to be the most appropriate Octave compatible solver.
\end{enumerate}

\pagebreak

\foreach \name in {ode45, ode23, ode113, ode15s, ode23s, ode23t, ode23tb}{
\subparagraph{\name} \ \\
\includegraphics[width=\linewidth]{stepFull\name} \\

\includegraphics[width=\linewidth]{stepDetail\name} \\

\includegraphics[width=\linewidth]{stepSS\name}
\pagebreak
}
\paragraph{MATLAB Code} \ \\
A simple \verb|getXdot| function is required to be passed into the ODE solver.

\begin{minted}[
		frame=lines,
		framesep=2mm,
		baselinestretch=1.2,
		bgcolor=gray!13,
		fontsize=\footnotesize,
		linenos,
		breaklines
		]{MATLAB}
function [ xdot ] = getXdot( t, x)
%getXdot return xdot from statespace for ODE45 use
%   t = filler variable
%   A = A matrix from system
%   x = initial state vector
%   B = B matrix from system
%   U = Input to system
global A B U
    xdot = A*x + B*U;
end	
\end{minted}


\begin{minted}[
		frame=lines,
		framesep=2mm,
		baselinestretch=1.2,
		bgcolor=gray!13,
		fontsize=\footnotesize,
		linenos,
		breaklines
		]{MATLAB}
%% test to use ode solvers to step PST-esq model
solverSelection = {'ode45', 'ode23', 'ode113','ode15s','ode23s','ode23t','ode23tb'}; % MATLAB
% 'ode15i' requires derivative at t=0... more thought required - availabe in octave aswell

%% pss model definition (miniWECC)
%           1   2   3   4       5   6   7       8       9       10
pss_con = [ 1  	1   20  2     0.25 0.04 0.2   0.03      1.0     -1.0];

%% MATLAB model - fixed step using lsim
tend = 60;

% PSS model creation
block1 = tf([pss_con(3)*pss_con(4), 0],[pss_con(4), 1]);
block2 = tf([pss_con(5), 1],[pss_con(6), 1]);
block3 = tf([pss_con(7), 1],[pss_con(8), 1]);
G= block1*block2*block3;

% lsim input
tL = 0:1/60/4:tend; % quarter cycle steps
modSig = zeros(size(tL,1),1);
modSig(tL>=1) = .001; % very small input to avoid limiter

% fixed step solution
yL = lsim(G,modSig,tL);

%% stock solver with using statespace system
% manipulate test sytem to statespace
[num,den] = tfdata(G);
global A B U
[A,B,C,D] = tf2ss(num{1},den{1});
    
    
for slnNum = 1:length(solverSelection)
    clear t1 t2 y1 y2
    odeName =solverSelection{slnNum}; % select ode function name from cell
    
    % Configure ODE settings
    %options = odeset('RelTol',1e-3,'AbsTol',1e-6); % default settings
    options = odeset('RelTol',1e-5,'AbsTol',1e-8,'InitialStep', 1/60/4, 'MaxStep',20);
    
    % initial conditions
    x = zeros(size(A,1),1);
    y0 = x;
    U = 0;
    
    % Pre-perturbance time interval solution
    [t1,y1] = feval(odeName, @getXdot, [0,1-1/60/4],y0, options); % feval used for variable ode solver selection
    yOut1 = C*y1'+D*U; % could be handled using 'outputfunction'
    
    % Step input
    U = modSig(end); % magnitude from fixed step inputs
    [t2,y2] = feval(odeName, @getXdot, [1,tend],y1(end,:)', options); % second interval solution
    yOut2 = C*y2'+D*U;
    
    % combining output from variable step solution
    tCombined = [t1;t2];
    yCombined = [yOut1, yOut2];
    
    % calculate step size
    tStep = zeros(length(tCombined),1);
    for tNdx = 2:length(tCombined)
        tStep(tNdx-1) = tCombined(tNdx)-tCombined(tNdx-1);
    end  
    
    % NOTE: Plotting code excluded
end
\end{minted}
\end{document}
