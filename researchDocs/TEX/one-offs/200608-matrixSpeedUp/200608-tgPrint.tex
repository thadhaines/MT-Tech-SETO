\documentclass[12pt]{article}
\usepackage[english]{babel}
\usepackage[utf8]{inputenc}

%% Pointer to 'default' preamble
% pacakages and definitions

\usepackage{geometry}
\geometry{
	letterpaper, 
	portrait, 
	top=.75in,
	left=.8in,
	right=.75in,
	bottom=.5in		} 	% Page Margins
	
%% additional packages for nice things
\usepackage{amsmath} 	% for most math
\usepackage{commath} 	% for abs
\usepackage{lastpage}	% for page count
\usepackage{amssymb} 	% for therefore
\usepackage{graphicx} 	% for image handling
\usepackage{wrapfig} 	% wrap figures
\usepackage[none]{hyphenat} % for no hyphenations
\usepackage{array} 		% for >{} column characterisctis
\usepackage{physics} 	% for easier derivative \dv....
\usepackage{tikz} 		% for graphic@!
\usepackage{circuitikz} % for circuits!
\usetikzlibrary{arrows.meta} % for loads
\usepackage[thicklines]{cancel}	% for cancels
\usepackage{xcolor}		% for color cancels
\usepackage[per-mode=fraction]{siunitx} % for si units and num
\sisetup{group-separator = {,}, group-minimum-digits = 3} % additional si unit table functionality

\usepackage{fancyhdr} 	% for header
\usepackage{comment}	% for ability to comment out large sections
\usepackage{multicol}	% for multiple columns using multicols
\usepackage[framed,numbered]{matlab-prettifier} % matlab sytle listing
\usepackage{marvosym} 	% for boltsymbol lightning
\usepackage{pdflscape} 	% for various landscape pages in portrait docs.
%\usepackage{float}
\usepackage{fancyvrb}	% for Verbatim (a tab respecting verbatim)
\usepackage{enumitem}	% for [resume] functionality of enumerate
\usepackage{spreadtab} 	% for using formulas in tables}
\usepackage{numprint}	% for number format in spread tab
\usepackage{subcaption} % for subfigures with captions
\usepackage[normalem]{ulem} % for strike through sout

% for row colors in tables....
\usepackage{color, colortbl}
\definecolor{G1}{gray}{0.9}
\definecolor{G2}{rgb}{1,0.88,1}%{gray}{0.6}
\definecolor{G3}{rgb}{0.88,1,1}

% For table formatting
\usepackage{booktabs}
\renewcommand{\arraystretch}{1.2}
\usepackage{floatrow}
\floatsetup[table]{capposition=top} % put table captions on top of tables

% Caption formating footnotesize ~ 10 pt in a 12 pt document
\usepackage[font={small}]{caption}

%% package config 
\sisetup{output-exponent-marker=\ensuremath{\mathrm{E}}} % for engineer E
\renewcommand{\CancelColor}{\color{red}}	% for color cancels
\lstset{aboveskip=2pt,belowskip=2pt} % for more compact table
%\arraycolsep=1.4pt\def
\setlength{\parindent}{0cm} % Remove indentation from paragraphs
\setlength{\columnsep}{0.5cm}
\lstset{
	style      = Matlab-editor,
	basicstyle = \ttfamily\footnotesize, % if you want to use Courier - not really used?
}
\renewcommand*{\pd}[3][]{\ensuremath{\dfrac{\partial^{#1} #2}{\partial #3}}} % for larger pd fracs
\renewcommand{\real}[1]{\mathbb{R}\left\{ #1 \right\}}	% for REAL symbol
\newcommand{\imag}[1]{\mathbb{I}\left\{ #1 \right\}}	% for IMAG symbol
\definecolor{m}{rgb}{1,0,1}	% for MATLAB matching magenta
	
%% custom macros
\newcommand\numberthis{\addtocounter{equation}{1}\tag{\theequation}} % for simple \numberthis command

\newcommand{\equal}{=} % so circuitikz can have an = in the labels
\newcolumntype{L}[1]{>{\raggedright\let\newline\\\arraybackslash\hspace{0pt}}m{#1}}
\newcolumntype{C}[1]{>{\centering\let\newline\\\arraybackslash\hspace{0pt}}m{#1}}
\newcolumntype{R}[1]{>{\raggedleft\let\newline\\\arraybackslash\hspace{0pt}}m{#1}}

%% Header
\pagestyle{fancy} % for header stuffs
\fancyhf{}
% spacing
\headheight 29 pt
\headsep 6 pt

%% Header
\rhead{Thad Haines \\ Page \thepage\ of \pageref{LastPage}}
\chead{tg Function Print \\ }
\lhead{Research \\ 6/8/20}

\usepackage{minted}

\begin{document}
%\begin{landscape}



\begin{minted}[
		frame=lines,
		framesep=2mm,
		baselinestretch=1.2,
		bgcolor=gray!13,
		fontsize=\footnotesize,
		linenos,
		breaklines
		]{MATLAB}
function tg(i,k,flag)
%TG simple turbine governor model init, network and differential solns
% Syntax: f = tg(i,k,flag)
%
% Input: i - generator number (0 for vector operation)
%        k - integer time
%        flag - 0 - initialization
%               1 - network interface computation
%               2 - system dynamics computation
%
% Output: 
%   NONE
%
% tg_con matrix format reference
%column	       data			unit
%  1	turbine model number (=1)
%  2	machine number
%  3	speed set point   wf		pu
%  4	steady state gain 1/R		pu
%  5	maximum power order  Tmax	pu on generator base
%  6	servo time constant   Ts	sec
%  7	governor time constant  Tc	sec
%  8	transient gain time constant T3	sec
%  9	HP section time constant   T4	sec
% 10	reheater time constant    T5	sec
%
%   History:
%   Date        Time    Engineer        Description
%   08/xx/93    xx:xx   Joe Chow        Version 1.0
%   (c) Copyright 1991-3 Joe H. Chow - All Rights Reserved
%   08/15/97    13:19   xxx             Version 1.x
%   06/05/20    10:19   Thad Haines     Revised format of globals and internal function documentation


global  mac_int pmech mac_spd

%global  tg_con tg_pot
%global  tg1 tg2 tg3 dtg1 dtg2 dtg3
%global  tg_idx n_tg tg_sig

global g


%jay = sqrt(-1);
if flag == 0 % initialization
    if i ~= 0
        if g.tg.tg_con(i,1) ~= 1
            error('TG: requires tg_con(i,1) = 1')
        end
    end
    if i ~= 0  % scalar computation
        n = mac_int(g.tg.tg_con(i,2)); % machine number
        
        % Check for pmech being inside generator limits
        if pmech(n,k) > g.tg.tg_con(i,5)
            error('TG init: pmech > upper limit, check machine base')
        end
        if pmech(n,k) < 0
            error('TG init: pmech < 0, check data')
        end
        
        % Initialize states
        g.tg.tg1(i,1) = pmech(n,k);
        %
        g.tg.tg_pot(i,1) = g.tg.tg_con(i,8)/g.tg.tg_con(i,7);
        a1 = 1 - g.tg.tg_pot(i,1);
        g.tg.tg_pot(i,2) = a1;
        g.tg.tg2(i,1) = a1*pmech(n,k);
        %
        g.tg.tg_pot(i,3) = g.tg.tg_con(i,9)/g.tg.tg_con(i,10);
        a2 = 1 - g.tg.tg_pot(i,3);
        g.tg.tg_pot(i,4) = a2;
        g.tg.tg3(i,1) = a2*pmech(n,k);
        %
        g.tg.tg_pot(i,5) = pmech(n,k);
        %
        g.tg.tg_sig(i,1)=0;
    else
        %  vectorized computation
        if g.tg.n_tg~=0
            n = mac_int(g.tg.tg_con(g.tg.tg_idx,2)); % machine number
            maxlmt = find(pmech(n,1) > g.tg.tg_con(g.tg.tg_idx,5));
            if ~isempty(maxlmt)
                n(maxlmt)
                error(' pmech excedes maximum limit')
            end
            minlmt = find(pmech(n,1) < zeros(g.tg.n_tg,1)); % min limit not user defined...
            if ~isempty(minlmt)
                n(minlmt)
                error('pmech less than zero')
            end
            g.tg.tg1(g.tg.tg_idx,1) = pmech(n,1);
            %
            g.tg.tg_pot(g.tg.tg_idx,1) = g.tg.tg_con(g.tg.tg_idx,8)./g.tg.tg_con(g.tg.tg_idx,7);
            a1 = ones(g.tg.n_tg,1) - g.tg.tg_pot(g.tg.tg_idx,1);
            g.tg.tg_pot(g.tg.tg_idx,2) = a1;
            g.tg.tg2(g.tg.tg_idx,1) = a1.*pmech(n,k);
            %
            g.tg.tg_pot(g.tg.tg_idx,3) = g.tg.tg_con(g.tg.tg_idx,9)./g.tg.tg_con(g.tg.tg_idx,10);
            a2 = ones(g.tg.n_tg,1) - g.tg.tg_pot(g.tg.tg_idx,3);
            g.tg.tg_pot(g.tg.tg_idx,4) = a2;
            g.tg.tg3(g.tg.tg_idx,1) = a2.*pmech(n,k);
            %
            g.tg.tg_pot(g.tg.tg_idx,5) = pmech(n,k);% set reference value
            g.tg.tg_sig(g.tg.tg_idx,1) = zeros(g.tg.n_tg,1);
        end
    end
end

if flag == 1 % network interface computation
    if i ~= 0 % scalar computation
        n = mac_int(g.tg.tg_con(i,2)); % machine number
        % the following update is needed because pmech depends on
        %   the output of the states tg1, tg2 and tg3
        pmech(n,k) = g.tg.tg3(i,k) + g.tg.tg_pot(i,3)*( g.tg.tg2(i,k) + g.tg.tg_pot(i,1)*g.tg.tg1(i,k) );
    else
        if g.tg.n_tg~=0
            n = mac_int(g.tg.tg_con(g.tg.tg_idx,2)); % machine number
            pmech(n,k) = g.tg.tg3(g.tg.tg_idx,k) + g.tg.tg_pot(g.tg.tg_idx,3).*( g.tg.tg2(g.tg.tg_idx,k) + g.tg.tg_pot(g.tg.tg_idx,1).*g.tg.tg1(g.tg.tg_idx,k) );
        end
    end
end

if flag == 2 % turbine governor dynamics calculation
    if i ~= 0 % scalar computation
        n = mac_int(g.tg.tg_con(i,2)); % machine number
        spd_err = g.tg.tg_con(i,3) - mac_spd(n,k);
        % addition of tg_sig
        demand = g.tg.tg_pot(i,5) + spd_err*g.tg.tg_con(i,4) + g.tg.tg_sig(i,k);
        demand = min( max(demand,0),g.tg.tg_con(i,5) ); % ensure limited demand
        % solve for derivative states
        g.tg.dtg1(i,k) = (demand - g.tg.tg1(i,k))/g.tg.tg_con(i,6);
        %
        g.tg.dtg2(i,k) = (g.tg.tg_pot(i,2)* g.tg.tg1(i,k)-g.tg.tg2(i,k))/g.tg.tg_con(i,7);
        %
        g.tg.dtg3(i,k) = ( (g.tg.tg2(i,k)+g.tg.tg_pot(i,1)*g.tg.tg1(i,k))*g.tg.tg_pot(i,4) - g.tg.tg3(i,k) )/ g.tg.tg_con(i,10);
        
        pmech(n,k) = g.tg.tg3(i,k) + g.tg.tg_pot(i,3)*(g.tg.tg2(i,k) + g.tg.tg_pot(:,1)*g.tg.tg1(i,k));
    else
        % vectorized computation
        if g.tg.n_tg ~=0
            n = mac_int(g.tg.tg_con(g.tg.tg_idx,2)); % machine number
            spd_err = g.tg.tg_con(g.tg.tg_idx,3) - mac_spd(n,k);
            demand = g.tg.tg_pot(g.tg.tg_idx,5) + spd_err.*g.tg.tg_con(g.tg.tg_idx,4) + g.tg.tg_sig(g.tg.tg_idx,k);
            demand = min( max(demand,zeros(g.tg.n_tg,1)),g.tg.tg_con(g.tg.tg_idx,5) );
            g.tg.dtg1(g.tg.tg_idx,k) = (demand - g.tg.tg1(g.tg.tg_idx,k))./g.tg.tg_con(g.tg.tg_idx,6);
            %
            g.tg.dtg2(g.tg.tg_idx,k) = ( g.tg.tg1(g.tg.tg_idx,k).*g.tg.tg_pot(g.tg.tg_idx,2) - g.tg.tg2(g.tg.tg_idx,k))./g.tg.tg_con(g.tg.tg_idx,7);
            %
            g.tg.dtg3(g.tg.tg_idx,k) = ((g.tg.tg2(g.tg.tg_idx,k) + g.tg.tg_pot(g.tg.tg_idx,1).*g.tg.tg1(g.tg.tg_idx,k)).*g.tg.tg_pot(g.tg.tg_idx,4) - g.tg.tg3(g.tg.tg_idx,k))./g.tg.tg_con(g.tg.tg_idx,10);
            
            pmech(n,k) = g.tg.tg3(g.tg.tg_idx,k) + g.tg.tg_pot(g.tg.tg_idx,3).*(g.tg.tg2(g.tg.tg_idx,k) + g.tg.tg_pot(g.tg.tg_idx,1).*g.tg.tg1(g.tg.tg_idx,k));
        end
    end
end


\end{minted}
%\end{landscape}
\end{document}
