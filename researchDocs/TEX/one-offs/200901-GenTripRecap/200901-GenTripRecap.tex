\documentclass[12pt]{article}
\usepackage[english]{babel}
\usepackage[utf8]{inputenc}

%% Pointer to 'default' preamble
% pacakages and definitions

\usepackage{geometry}
\geometry{
	letterpaper, 
	portrait, 
	top=.75in,
	left=.8in,
	right=.75in,
	bottom=.5in		} 	% Page Margins
	
%% additional packages for nice things
\usepackage{amsmath} 	% for most math
\usepackage{commath} 	% for abs
\usepackage{lastpage}	% for page count
\usepackage{amssymb} 	% for therefore
\usepackage{graphicx} 	% for image handling
\usepackage{wrapfig} 	% wrap figures
\usepackage[none]{hyphenat} % for no hyphenations
\usepackage{array} 		% for >{} column characterisctis
\usepackage{physics} 	% for easier derivative \dv....
\usepackage{tikz} 		% for graphic@!
\usepackage{circuitikz} % for circuits!
\usetikzlibrary{arrows.meta} % for loads
\usepackage[thicklines]{cancel}	% for cancels
\usepackage{xcolor}		% for color cancels
\usepackage[per-mode=fraction]{siunitx} % for si units and num
\sisetup{group-separator = {,}, group-minimum-digits = 3} % additional si unit table functionality

\usepackage{fancyhdr} 	% for header
\usepackage{comment}	% for ability to comment out large sections
\usepackage{multicol}	% for multiple columns using multicols
\usepackage[framed,numbered]{matlab-prettifier} % matlab sytle listing
\usepackage{marvosym} 	% for boltsymbol lightning
\usepackage{pdflscape} 	% for various landscape pages in portrait docs.
%\usepackage{float}
\usepackage{fancyvrb}	% for Verbatim (a tab respecting verbatim)
\usepackage{enumitem}	% for [resume] functionality of enumerate
\usepackage{spreadtab} 	% for using formulas in tables}
\usepackage{numprint}	% for number format in spread tab
\usepackage{subcaption} % for subfigures with captions
\usepackage[normalem]{ulem} % for strike through sout

% for row colors in tables....
\usepackage{color, colortbl}
\definecolor{G1}{gray}{0.9}
\definecolor{G2}{rgb}{1,0.88,1}%{gray}{0.6}
\definecolor{G3}{rgb}{0.88,1,1}

% For table formatting
\usepackage{booktabs}
\renewcommand{\arraystretch}{1.2}
\usepackage{floatrow}
\floatsetup[table]{capposition=top} % put table captions on top of tables

% Caption formating footnotesize ~ 10 pt in a 12 pt document
\usepackage[font={small}]{caption}

%% package config 
\sisetup{output-exponent-marker=\ensuremath{\mathrm{E}}} % for engineer E
\renewcommand{\CancelColor}{\color{red}}	% for color cancels
\lstset{aboveskip=2pt,belowskip=2pt} % for more compact table
%\arraycolsep=1.4pt\def
\setlength{\parindent}{0cm} % Remove indentation from paragraphs
\setlength{\columnsep}{0.5cm}
\lstset{
	style      = Matlab-editor,
	basicstyle = \ttfamily\footnotesize, % if you want to use Courier - not really used?
}
\renewcommand*{\pd}[3][]{\ensuremath{\dfrac{\partial^{#1} #2}{\partial #3}}} % for larger pd fracs
\renewcommand{\real}[1]{\mathbb{R}\left\{ #1 \right\}}	% for REAL symbol
\newcommand{\imag}[1]{\mathbb{I}\left\{ #1 \right\}}	% for IMAG symbol
\definecolor{m}{rgb}{1,0,1}	% for MATLAB matching magenta
	
%% custom macros
\newcommand\numberthis{\addtocounter{equation}{1}\tag{\theequation}} % for simple \numberthis command

\newcommand{\equal}{=} % so circuitikz can have an = in the labels
\newcolumntype{L}[1]{>{\raggedright\let\newline\\\arraybackslash\hspace{0pt}}m{#1}}
\newcolumntype{C}[1]{>{\centering\let\newline\\\arraybackslash\hspace{0pt}}m{#1}}
\newcolumntype{R}[1]{>{\raggedleft\let\newline\\\arraybackslash\hspace{0pt}}m{#1}}

%% Header
\pagestyle{fancy} % for header stuffs
\fancyhf{}
% spacing
\headheight 29 pt
\headsep 6 pt

%% Header
\rhead{Thad Haines \\ Page \thepage\ of \pageref{LastPage}}
\chead{Collection of Generator Tripping \\ and Un-Tripping Information  }
\lhead{Research \\ 09/01/20}

\usepackage[hidelinks]{hyperref} % allow links in pdf
\usepackage{setspace}
\usepackage{multicol}
\usepackage{lscape}
%\usepackage{minted}

\begin{document}
\onehalfspacing
%\raggedright
\paragraph{Document Purpose / Intent} \ \\
An extended term case may require the insertion of additional modeled generation resources.
There was no known method in PST to increase inertial generation during a simulation and the method used for removing a generator was unclear.
This document is meant to describe initial work that has been done to develop a method to 'un-trip' a generator, as well as describe how a generator is tripped in PST.
Various code additions and possible future work are also presented.


\paragraph{How PST Trips Generators} \ \\
During simulation initialization, \verb|g.mac.mac_trip_flags| is initialized as a column vector of zeros that correspond to the \verb|mac_con| array, and 
\verb|g.mac.mac_trip_states| variable is set to zero. %\verb|g.mac.mac_trip_states|  ( appears unused).
To trip a generator, the \verb|mac_trip_flag| corresponding to the desired generator it trip is set to $1$ via the user generated \verb|mac_trip_logic| code.
The \verb|mac_trip_logic| is executed in the \verb|initStep| function which alters \verb|g.mac.mac_trip_flags| to account for any programmed trips.
Specifically, a $0$ in the \verb|g.mac.mac_trip_flags| row vector is changed to a 1 to signify a generator has tripped.\\


The \verb|g.mac.mac_trip_flags| vector is summed in the \verb|networkSolution| (\verb|networkSoltuionVTS|) function.
If the resulting sum is larger than 0.5, the line number connected to the generator in \verb|g.line.line_sim| is found and the reactance is set to infinity (1e7).
The reduced y matrices are then recalculated and used to solve the network solution via an \verb|i_simu| call.
If derivatives of the tripped machine are not set to zero, the generator's speed increases, mechanical power output eventually drops to zero/near-zero, and the attached exciter Efd appears to the set reference voltage.

%Realistically, Pmech and all P and Q limits should also probably be set to zero which may `clean up' values.

\subparagraph{Performance Note} \ \\
If a machine is tripped, the altered reduced Y matrices are generated every simulation step.
This repeated action could probably be reduced or eliminated via use of globals and logic checks.
\begin{itemize}
\item create new globals handling the Y matrices selected during the network solution.
\item create an \verb|old_mac_trip_flags|
\item compare \verb|old_mac_trip_flags| to \verb|mac_trip_flags|
\item if there has been a new trip, update associated Y matrix globals
\item if there hasn't been a new trip, but a machine has tripped, use the stored Y matrix
\item must account for other system changes due to faulting scenarios
\end{itemize} 
This is merely a calculation reduction and not required for tripping/un-tripping, but may result in a noticeable speed up.

\pagebreak
\paragraph{Experimental 'Un-trip' Method} \ \\
A method was devised to bring generation online that involves changing the \verb|mac_trip_flag| from 1 to 0, re-initializing or bypassing models, and  ramping exciter and governor reference values.

Generally, the current method involves:
\begin{enumerate}
\itemsep 0 em
\item Ensure $P_{mech} = P_{elect} = 0$
\item `Bypass' governor by setting $P_{ref} = 0$ and $\omega_{ref} = $ the current machine speed.
\item `Bypass' Exciter by ignoring calculated \verb|mac.vex| from exciter model. (may change)
\item Change \verb|mac_trip_flag| from 1 to 0
\item Re-Initialize machine to current time index
\item Re-Initialize governor to current time index
\item Re-Initialize exciter and PSS to current time index
\item Ramp exciter and governor reference values
\end{enumerate}

Changing a \verb|mac_trip_flag| from 1 to 0 returns the line reactance to normal and effectively re-connects a generator to the system.
However, the model states and reference values are no longer synchronized with the system and must be handled to avoid unrealistic state scenarios and excessive transients.
Test cases have been created that utilize the  \verb|mac_trip_logic| and \verb|mtg_sig| user defined code files to manage the required bypassing, reinitializations, and reference ramping tasks.\\


Most PST models may be initialized if the passed in flag is 0.
Individual models may be intialized if the passed in $i$ is the internal number of the model to initialize.
Unfortunately, the initialize functions were designed to work with values at time index 1 however, it appears simple enough to change this so that an arbitrary data index may be referenced for initialization procedures.
Alternatively, additional functions, or flag functions of models may be created to handle re-initialization.\\


\begin{minipage}[t]{0.47\linewidth}
\raggedright
\footnotesize
\paragraph{Added Code Functionality}
\begin{itemize}
\itemsep 0 em
\item Exciter Bypass
\item Zeroing of tripped Machine derivatives in FTS mode
\item Update of PSS initialization routine to handle re-init.
\item Creation of re-init functions for \verb|mac_sub|, \verb|tg|, \verb|smpexc|.
\end{itemize}
\vfill \null
\end{minipage}%
\begin{minipage}[t]{0.47\linewidth}
\raggedright
\footnotesize
\paragraph{Possible Future Work}
\begin{itemize}
\itemsep 0 em
\item Initialize system with tripped generators\\ (i.e. allow spinning reserve)
\item Reduce Y matix redundant calculations when tripping via globals and logic
\item Enable re-init for more generators/exciters
\item Automatically zero derivatives of models attached to gen (exc, pss, gov)
\item Devise better, more functionalized, method to more easily trip and un-trip generators 
\item Allow VTS to work with un-trips - will probably require indexing of saved solution vector.
\end{itemize}
\end{minipage}%



\end{document}
