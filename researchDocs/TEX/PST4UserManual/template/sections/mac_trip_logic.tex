\section{Machine Trip Logic (mac\_trip\_logic)}  
A \verb|mac_trip_logic| file is created by a user and placed in the root directory of PST to control the tripping of machines.
This added functionality allows for multiple generator trips during a single simulation.
The procedure PST performs to accomplish such a task is as follows:

During simulation initialization, \verb|g.mac.mac_trip_flags| is initialized as a column vector of zeros that correspond to the \verb|mac_con| array, and 
\verb|g.mac.mac_trip_states| variable is set to zero. %\verb|g.mac.mac_trip_states|  ( appears unused).
To trip a generator, the \verb|mac_trip_flag| corresponding to the desired generator to trip is set to $1$ via the user generated \verb|mac_trip_logic| code.
\verb|mac_trip_logic| is executed in the \verb|initStep| function which alters \verb|g.mac.mac_trip_flags| to account for any programmed trips.
Specifically, a $0$ in the \verb|g.mac.mac_trip_flags| vector is changed to a 1 to signify a generator has tripped.


The \verb|g.mac.mac_trip_flags| vector is summed in the \verb|networkSolution| or \\ \verb|networkSoltuionVTS|  function.
If the resulting sum is larger than 0.5, the line number connected to the generator in \verb|g.line.line_sim| is found and the reactance is set to infinity (1e7).
The reduced Y matrices are then recalculated and used to solve the network solution via an \verb|i_simu| call.
It should be noted that if a machine is tripped, the altered reduced Y matrices are generated every simulation step.
This repeated action could be minimized via use of globals and logic checks.