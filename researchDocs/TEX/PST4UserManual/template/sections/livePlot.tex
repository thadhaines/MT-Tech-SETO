\section{Live Plotting Functionality (liveplot)} 
Default action of PST 3 was to plot the bus voltage magnitude at a faulted bus.
However, it may be useful to plot other values or turn off the plotting during a simulation.
The live plotting routine is now functionalized to allow users to more easily define what is displayed (if anything) during simulations.
The \verb|livePlot| function was designed to be overwritten during batch simulation runs as shown in Listing \ref{lst: liveplot example}.

\begin{lstlisting}[caption={Live Plotting Overwrite Example},label={lst: liveplot example}]
\end{lstlisting}\vspace{-2 em}
\begin{minted}[
		frame=lines,
		framesep=2mm,
		baselinestretch=1.2,
		bgcolor=gray!13,
		fontsize=\footnotesize,
		%linenos,
		breaklines
		]{MATLAB}
% PSTpath is the location of the root PST directory
copyfile([PSTpath 'liveplot_2.m'],[PSTpath 'liveplot.m']); % Plot AGC signals 
copyfile([PSTpath 'liveplot_ORIG.m'],[PSTpath 'liveplot.m']); % restore functionality
\end{minted}

There are currently 3 live plot functions:

\begin{itemize}
\itemsep 0 em
\item \verb|liveplot_ORIG| - Original faulted bus voltage plot.
\item \verb|liveplot_1| - Faulted bus voltage and system machine speeds plus any lmod signals.
\item \verb|liveplot_2| - AGC signals.
\end{itemize}

It should be noted that the live plotting can cause extremely slow simulations and occasional crashes.
To disable live plotting:
\begin{itemize}
\itemsep 0 em
\item In stand-alone mode: \\Change the `live plot?' field from a $1$ to a $0$ in the popup dialog box.
\item In batch mode:  \\Create a variable \verb|livePlotFlag| and set as $0$ or \verb|false|.
\end{itemize}