\section{Area Definitions}  
Lacking in previous versions of PST was a formal way to define areas.
While many examples were commonly categorized as multi-area, they were handled the same as single area systems by PST.

PST 4 allows a user to create areas via an \verb|area_def| array defined in a system data file.
An example of a two area \verb|area_def| array is shown in Listing \ref{lst: area def}.

\begin{lstlisting}[caption={Area Definition Example},label={lst: area def}]
\end{lstlisting}\vspace{-2 em}
\begin{minted}[
		frame=lines,
		framesep=2mm,
		baselinestretch=1.2,
		bgcolor=gray!13,
		fontsize=\footnotesize,
		%linenos,
		breaklines
		]{MATLAB}
%% area_def data format
% NOTE: should contain same number of rows as bus array (i.e. all bus areas defined)
% col 1 bus number
% col 2 area number
area_def = [ ...
            1  1;
            2  1;
            3  1;
            4  1;
            10 1;
            11 2;
            12 2;
            13 2;
            14 2; 
            20 1;
           101 1; 
           110 2;
           120 2];
\end{minted}

Created areas are stored in the structured global (see Section \ref{ss: area globals}) and track interchange, average frequency, and area inertia.
An area is required for the AGC model to operate (see Section \ref{sec: AGC def}).

