\vspace{2em} % for tech wonkyness
\chapter*{Power System Toolbox Introduction}
\addcontentsline{toc}{\fmAddAs}{Power System Toolbox Introduction}

% Note the required use of \vspace{1em} to seperate paragraphs.

Power System Toolbox (PST) is MATLAB code used to: 1) solve power flow problems; 2) simulate power-system transients; and 3) to linearize a power system model.  All code is open-access and free.  PST is widely used by researchers studying problems related to power-system dynamics.  Its open access format enables researchers to customize models and functions to meet their unique needs.  Being a MATLAB based code, one can incorporate powerful MATLAB functions in into their problem solving process.

\vspace{1em}
PST was the brain child of Dr. Joe Chow of Rensselaer Polytechnic Institute with the original version written by him and Dr. Kwok W. Cheung in the early 1990s.  The late Mr. Graham Rogers made significant contributions and included it in his book \cite{rogers1999}.  Many others have added customized models and variations over the past two plus decades. 

\vspace{1em}
The basis for the version of PST presented in this document is version 3, available from Dr. Chow’s webpage (\href{https://www.ecse.rpi.edu/~chowj/}{https://www.ecse.rpi.edu/~chowj/}) with added customizable current-injection functions written by Dr. Dan Trudnowski from Montana Technological University.