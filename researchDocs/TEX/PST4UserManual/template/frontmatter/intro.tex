\vspace{2em} % for tech wonkyness
\chapter*{Introduction  - WIP}
%\addcontentsline{toc}{\fmAddAs}{\tocAbstract}%%% All caps == ugly
% Note the required use of \vspace{1em} to seperate paragraphs.

This paragraph should contain a bit of history about how Power System Toolbox (PST) has developed into its current form- Possibly written/drafted by Trudnowski...
Mention of renewed interest in PST via SETO/Sandia and the new MIT license may be appropriate.

\vspace{1em}
%A brief explanation of the user manual purpose and major content would make sense.
The purpose of this user manual is to document work done on PST 3 to form PST 4. 
It is meant only to augment the previous user manual provided with \mbox{PST 3} \cite{PST3manual}.
%
Major code changes presented include 
how global variables are handled
and the
functionalization of the non-linear simulation routine.
%
Descriptions of new models created for
inverter based resources
%power (or current) injection, 
%a voltage source behind a reactance,
and
automatic generation control
are also presented.
%
Additionally, work to add functionality to PST such as
variable time step integration routines,
generator tripping, % mention un-tripping
and
code compatibility with Octave
is documented.
%
To demonstrate and debug new and old PST capabilities, an example library has been created and brief explanations of examples are intended to be included.
Unfortunately, due to time constraints, details may be rather lacking and intentions may only matter so much.

\vspace{1em}
Finally, it is worth noting that all source code, examples, and other research documentation can be accessed at 
\href{https://github.com/thadhaines/MT-Tech-SETO/tree/master/PST}{https://github.com/thadhaines/MT-Tech-SETO/tree/master/PST}.