\vspace{2em} % for tech wonkyness
\chapter*{User Manual Introduction}
\addcontentsline{toc}{\fmAddAs}{User Manual Introduction} %%% ugly = ALL CAPS
%\addcontentsline{toc}{\fmAddAs}{\tocAbstract}%%% All caps == ugly
% Note the required use of \vspace{1em} to seperate paragraphs.

%A brief explanation of the user manual purpose and major content would make sense.
The purpose of this user manual is to document work done on PST 3 to form PST 4. 
It is meant only to augment the previous user manuals and other available documentation about \mbox{PST} \cite{chow1992, PST2LFTut, PST3manual, chow2015}.
%
Major code changes presented include 
how global variables are handled
and the
functionalization of the non-linear simulation routine.

\vspace{1em}
Descriptions of new models created for
inverter based resources
%power (or current) injection, 
%a voltage source behind a reactance,
and
automatic generation control
are also presented.
Additionally, work to add functionality to PST such as
variable time step integration routines,
generator tripping, % mention un-tripping
and
code compatibility with Octave
is documented.



\vspace{1em}
To demonstrate and debug new and old PST capabilities, 
an example library has been created and brief results of select examples is included in this document.
Unfortunately, due to time constraints, details are likely lacking.
Sections in this document that are unfinished are denoted by a WIP in the header.
Despite lack of documentation,  the code examples have been checked for functionality and \emph{should} work as designed.

%and intentions may only matter so much.

\vspace{1em}
Finally, it is worth noting that all source code, examples, and other research documentation can be accessed at 
\href{https://github.com/thadhaines/MT-Tech-SETO/tree/master/PST}{https://github.com/thadhaines/MT-Tech-SETO/tree/master/PST}.

% PST Introduction, PWRMOD, and IVMMOD information adapted from Dr. Dan Trudnowski