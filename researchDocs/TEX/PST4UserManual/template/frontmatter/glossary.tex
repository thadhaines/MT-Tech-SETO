\chapter*{Glossary of Terms}
\addcontentsline{toc}{\fmAddAs}{\tocGlossary} %%%%
\vspace{-2em}
% There are may be a better way to do this (i.e. use of glossaries package)
% Obviously, A table is used for now.

% using the glossaries package requires the installation of perl.
% additionally, the glossaries package reuires a seperate build of the glossary so that t prints at all (similar to the bibliography using biber)
\renewcommand*{\glsclearpage}{} % remove pagebreak pre-glossary
\renewcommand{\glossarysection}[2][]{} % remove auto glossary title

\newglossaryentry{IPY}{name={IPY},%
	description={ IronPython }}
\newglossaryentry{PY3}{name={PY3},%
	description={ Python version 3.x}}
\newglossaryentry{AMQP}{name={AMQP},%
	description={Advanced Message Queue Protocol}}
\newglossaryentry{ACE}{name={ACE},%
	description={Area Control Error }}
\newglossaryentry{AC}{name={AC},%
	description={Alternating Current}}
\newglossaryentry{PU}{name={PU},%
	description={Per-Unit}}
\newglossaryentry{SACE}{name={SACE},%
	description={Smoothed ACE}}
\newglossaryentry{IACE}{name={IACE},%
	description={Integral of ACE}}
\newglossaryentry{TLB}{name={TLB},%
	description={Tie-Line Bias}}
\newglossaryentry{ERCOT}{name={ERCOT},%
	description={Electric Reliability Council of Texas }}
\newglossaryentry{PI}{name={PI},%
	description={Proportional and Integral}}
\newglossaryentry{PST}{name={PST},%
	description={Power System Toolbox}}
\newglossaryentry{RACE}{name={RACE},%
	description={Reported ACE}}
\newglossaryentry{DACE}{name={DACE},%
	description={Distributed ACE}}
\newglossaryentry{PSLF }{name={PSLF},%
	description={Positive Sequence Load Flow}}
\newglossaryentry{PSDS}{name={PSDS},%
	description={PSLF Dynamic Subsystem}}
\newglossaryentry{PSS}{name={PSS},%
	description={Power System Stabilizer }}
\newglossaryentry{AGC}{name={AGC},%
	description={Automatic Generation Control }}
\newglossaryentry{LFC}{name={LFC},%
	description={Load Frequency Control }}
\newglossaryentry{WECC}{name={WECC},%
	description={Western Electricity Coordinating Council }}
\newglossaryentry{NERC}{name={NERC},%
	description={North American Electric Reliability Corporation }}
\newglossaryentry{FERC }{name={FERC},%
	description={Federal Energy Regulatory Commission }}
\newglossaryentry{PSLTDSim}{name={PSLTDSim},%
	description={Power System Long-Term Dynamic Simulator}}
\newglossaryentry{LTD}{name={LTD},%
	description={ Long-Term Dynamic}}
\newglossaryentry{API}{name={API},%
		description={ Application Programming Interface }}
\newglossaryentry{EIA}{name={EIA},%
		description={ United States Energy Information Administration}}
\newglossaryentry{PyPI}{name={PyPI},%
		description={ Python Package Index}}
\newglossaryentry{TSPF}{name={TSPF},%
		description={ Time-Sequenced Power Flow}}
\newglossaryentry{CTS}{name={CTS},%
		description={ Classical Transient Stability}}
\newglossaryentry{ODE}{name={ODE},%
	description={ Ordinary Differential Equation }}
\newglossaryentry{FTL}{name={FTL},%
	description={ Frequency Trigger Limit }}
\newglossaryentry{IPC}{name={IPC},%
	description={ Interprocess Communication }}
\newglossaryentry{DTC}{name={DTC},%
	description={ Definite Time Controller }}
\newglossaryentry{CLR}{name={CLR},%
	description={ Common Language Runtime }}
\newglossaryentry{PMIO}{name={PMIO},%
	description={ PSLF Model Information Object }}
\newglossaryentry{US}{name={US},%
	description={United States of America }}
\newglossaryentry{RTO}{name={RTO},%
	description={Regional Transmission Organization}}
\newglossaryentry{ISO}{name={ISO},%
	description={Independent Service Operator}}
\newglossaryentry{SI}{name={SI},%
	description={International System of Units}}
\newglossaryentry{Hz}{name={Hz},%
	description={Hertz, cycles per second}}
\newglossaryentry{W}{name={W},%
	description={Watt, Joules per second}}
\newglossaryentry{J}{name={J},%
	description={Joule, Neton meters, Watt seconds}}
\newglossaryentry{ABS}{name={ABS},%
	description={Agent Based Simulation}}
\newglossaryentry{BES}{name={BES},%
	description={Bulk Electrical System}}
\newglossaryentry{BA}{name={BA},%
	description={Balancing Authority}}
\newglossaryentry{P}{name={P},%
	description={Real Power}}
\newglossaryentry{Q}{name={Q},%
	description={Reactive power}}
\newglossaryentry{VAR}{name={VAR},%
	description={Volt Amps Reactive}}
\newglossaryentry{GE}{name={GE},%
	description={General Electric}}
\newglossaryentry{ABM}{name={ABM},%
	description={Agent Based Modeling}}
\newglossaryentry{IC}{name={IC},%
	description={Interchange}}
\glsaddall % so that there is no need to call \gls{label} for each term
\printglossaries


\begin{comment}
% table style glossary for record...

\begin{table}[h]
	\begin{tabular}{@{} p{.25\linewidth} p{.7\linewidth} @{}} %\toprule 
	\textbf{Term} & \textbf{Definition}\\
	%	
	\LaTeX 			& A document preparation system. Favors logical design over visual design. Reportedly widely used in academia... \\
	PSLF 	&	Positive Sequence Load Flow. GE's power system simulation software.\\
	PSDS	&	PSLF Dynamic subsystem.\\
	PSS		&	Power System Stabilizer \\
	AGC		&	Automatic Generation Control \\
	LFC		&	Load Frequency Control \\
	WECC	&	Western Electricity Coordinating Council \\
	NERC	&	North American Electric Reliability Corporation \\
	FERC 	&	Federal Energy Regulatory Commission \\
	PSLTDSim &	Power System Long-Term Dynamic Simulator\\
	LTD		& 	Long-Term Dynamic. May also refer the type of simulation performed by PSLTDSim.\\

	\end{tabular}
\end{table}

\end{comment}
