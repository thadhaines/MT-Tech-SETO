\chapter*{Glossary of Terms}
\addcontentsline{toc}{\fmAddAs}{\tocGlossary} %%%%
\vspace{-2em}
% There are may be a better way to do this (i.e. use of glossaries package)
% Obviously, A table is used for now.

% using the glossaries package requires the installation of perl.
% additionally, the glossaries package reuires a seperate build of the glossary so that t prints at all (similar to the bibliography using biber)
\renewcommand*{\glsclearpage}{} % remove pagebreak pre-glossary
\renewcommand{\glossarysection}[2][]{} % remove auto glossary title

\newglossaryentry{ACE}{name={ACE}, description={Area Control Error }}
\newglossaryentry{AC}{name={AC}, description={Alternating Current}}
\newglossaryentry{DC}{name={DC}, description={Direct Current}}
\newglossaryentry{PU}{name={PU}, description={Per-Unit}}
\newglossaryentry{SACE}{name={SACE}, description={Smoothed ACE}}
\newglossaryentry{PI}{name={PI}, description={Proportional and Integral}}
\newglossaryentry{PST}{name={PST}, description={Power System Toolbox}}
\newglossaryentry{RACE}{name={RACE}, description={Reported ACE}}
\newglossaryentry{DACE}{name={DACE}, description={Distributed ACE}}
\newglossaryentry{PSS}{name={PSS}, description={Power System Stabilizer }}
\newglossaryentry{AGC}{name={AGC}, description={Automatic Generation Control }}
\newglossaryentry{WECC}{name={WECC}, description={Western Electricity Coordinating Council }}
\newglossaryentry{NERC}{name={NERC}, description={North American Electric Reliability Corporation }}
\newglossaryentry{FERC }{name={FERC}, description={Federal Energy Regulatory Commission }}
\newglossaryentry{EIA}{name={EIA}, description={ United States Energy Information Administration}}
%\newglossaryentry{CTS}{name={CTS}, description={ Classical Transient Stability}}
\newglossaryentry{ODE}{name={ODE}, description={ Ordinary Differential Equation }}
%\newglossaryentry{US}{name={US}, description={United States of America }}
%\newglossaryentry{RTO}{name={RTO}, description={Regional Transmission Organization}}
%\newglossaryentry{ISO}{name={ISO}, description={Independent Service Operator}}
%\newglossaryentry{SI}{name={SI}, description={International System of Units}}
\newglossaryentry{Hz}{name={Hz}, description={Hertz, cycles per second}}
\newglossaryentry{W}{name={W}, description={Watt, Joules per second}}
\newglossaryentry{J}{name={J}, description={Joule, Neton meters, Watt seconds}}
%\newglossaryentry{BES}{name={BES}, description={Bulk Electrical System}}
%\newglossaryentry{BA}{name={BA}, description={Balancing Authority}}
\newglossaryentry{P}{name={P}, description={Real Power}}
\newglossaryentry{Q}{name={Q}, description={Reactive power}}
\newglossaryentry{VAR}{name={VAR}, description={Volt Amps Reactive}}
%\newglossaryentry{IC}{name={IC}, description={Interchange}}
\newglossaryentry{VTS}{name={VTS}, description={Variable Time Step}}
\newglossaryentry{FTS}{name={FTS}, description={Fixed Time Step}}

\glsaddall % so that there is no need to call \gls{label} for each term
\printglossaries


\begin{comment}
% table style glossary for record...

\begin{table}[h]
	\begin{tabular}{@{} p{.25\linewidth} p{.7\linewidth} @{}} %\toprule 
	\textbf{Term} & \textbf{Definition}\\
	%	
	\LaTeX 			& A document preparation system. Favors logical design over visual design. Reportedly widely used in academia... \\
	PSLF 	&	Positive Sequence Load Flow. GE's power system simulation software.\\
	PSDS	&	PSLF Dynamic subsystem.\\
	PSS		&	Power System Stabilizer \\
	AGC		&	Automatic Generation Control \\
	LFC		&	Load Frequency Control \\
	WECC	&	Western Electricity Coordinating Council \\
	NERC	&	North American Electric Reliability Corporation \\
	FERC 	&	Federal Energy Regulatory Commission \\
	PSLTDSim &	Power System Long-Term Dynamic Simulator\\
	LTD		& 	Long-Term Dynamic. May also refer the type of simulation performed by PSLTDSim.\\

	\end{tabular}
\end{table}

\end{comment}
