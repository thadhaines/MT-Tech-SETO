% file to contain information on general operation of v4
% s_simu, flochart, linear sim....
%==============================================================================
\chapter{PST Operation Overview}

This is likely to include a flow chart/ multiple flow charts.

Mention partitioned explicit method via Huen's, partitioned implicit via VTS.

Overview of running \verb|s_simu| in batch or stand alone mode.

copying of modulation files

%==============================================================================
\section{Simulation Loop}
The complete simulation loop code is shown below.
This code was copied from \verb|s_simu_BatchVTS| with corresponding line numbers.

\begin{minted}[
		frame=lines,
		framesep=2mm,
		baselinestretch=1.2,
		bgcolor=gray!13,
		fontsize=\footnotesize,
		linenos,
		firstnumber=362,
		breaklines
				]{MATLAB}
%% Simulation loop start
warning('*** Simulation Loop Start')
for simTblock = 1:size(g.vts.t_block)
    
    g.vts.t_blockN = simTblock;
    g.k.ks = simTblock; % required for huen's solution method.
    
    if ~isempty(g.vts.solver_con)
        odeName = g.vts.solver_con{g.vts.t_blockN};
    else
        odeName = 'huens'; % default PST solver
    end
    
    if strcmp( odeName, 'huens')
        % use standard PST huens method
        fprintf('*** Using Huen''s integration method for time block %d\n*** t=[%7.4f, %7.4f]\n', ...
            simTblock, g.vts.fts{simTblock}(1), g.vts.fts{simTblock}(end))
        
        % add fixed time vector to system time vector
        nSteps = length(g.vts.fts{simTblock});
        g.sys.t(g.vts.dataN:g.vts.dataN+nSteps-1) = g.vts.fts{simTblock};
        
        % account for pretictor last step time check
        g.sys.t(g.vts.dataN+nSteps) = g.sys.t(g.vts.dataN+nSteps-1)+ g.sys.sw_con(simTblock,7);
        
        for cur_Step = 1:nSteps
            k = g.vts.dataN;
            j = k+1;
            
            % display k and t at every first, last, and 50th step
            if ( mod(k,50)==0 ) || cur_Step == 1 || cur_Step == nSteps
                fprintf('*** k = %5d, \tt(k) = %7.4f\n',k,g.sys.t(k)) % DEBUG
            end
            
            %% Time step start
            initStep(k)
            
            %% Predictor Solution =========================================
            networkSolutionVTS(k, g.sys.t(k))
            monitorSolution(k);
            dynamicSolution(k)
            dcSolution(k)
            predictorIntegration(k, j, g.k.h_sol)   % g.k.h_sol updated i_simu
            
            %% Corrector Solution =========================================
            networkSolutionVTS(j, g.sys.t(j))
            dynamicSolution(j)
            dcSolution(j)
            correctorIntegration(k, j, g.k.h_sol)
            
            % most recent network solution based on completely calculated states is k
            monitorSolution(k);
            %% Live plot call
            if g.sys.livePlotFlag
                livePlot(k)
            end
            
            g.vts.dataN = j;                        % increment data counter
            g.vts.tot_iter = g.vts.tot_iter  + 2;   % increment total solution counter
            g.vts.slns(g.vts.dataN) = 2;            % track step solution
        end
        % Account for next time block using VTS
        handleStDx(j, [], 3) % update g.vts.stVec to initial conditions of states
        handleStDx(k, [], 1) % update g.vts.dxVec to initial conditions of derivatives 
        
    else % use given variable method
        fprintf('*** Using %s integration method for time block %d\n*** t=[%7.4f, %7.4f]\n', ...
            odeName, simTblock, g.vts.t_block(simTblock, 1), g.vts.t_block(simTblock, 2))
        
        % feval used for ODE call - could be replaced with if statements.
        feval(odeName, inputFcn, g.vts.t_block(simTblock,:), g.vts.stVec , options);
        
        % Alternative example of using actual function name:
        %ode113(inputFcn, g.vts.t_block(simTblock,:), g.vts.stVec , options);
    end
    
end% end simulation loop
\end{minted}