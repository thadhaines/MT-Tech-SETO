% file to contain information on loose ends or things generaly unfinished
%==============================================================================
\chapter{Loose Ends - WIP}
As software development is never actually `done', this chapter is meant to contain any loose ends that felt relevant.
%The below is copied from the most recent weekly (09/02/20) and some minor additions

\begin{enumerate}
\singlespacing
	\item As infinite buses don't seem to be used in dynamic simulation, they were not converted to use the golbal g.
	\item \verb|tgh| model was not converted for use with global g. (no examples of tgh gov)
	\item In original (and current) \verb|s_simu|, the global \verb|tap| value associated with HVDC is over-written with  a value used to compute line current multiple times. \\It probably shouldn't be.
	\item Constant Power or Current loads seem to require a portion of constant Impedance.
	\item PSS design functionality not explored.
	\item No examples of of delta P omega filter or user defined damping controls for SVC and TCSC models
	\item Differences in \verb|mac_ind| between PST 2 and 3 seem backward compatible, but this is untested.
	\item DC is not implemented in VTS. It seems like DC models should be combined into the main routine if so desired. Seems counter intuitive / (not very possible) to do multi-rate variable time step integration...
	\item AGC capacity should consider defined machine limits instead of assuming 1 PU max.
	\item AGC should allow for a `center of inertia' frequency option instead of the weighted average frequency.
	\item A method to initialize the power system with tripped generators should be devised and occur before the first power flow solution.
	\item A method to zero derivatives of any model attached to a tripped generator should be created to enable VTS to optimize time steps.
	\item Re-initializing a tripped generator in VTS will likely require indexing the \verb|g.vts.stVec|. This could be aided by adding indices to the \verb|g.vts.fsdn| cell.
	\item miniWECC DC lines (modeled as power injection) are not included in AGC calculations as the power does not travel over any simulated lines.
	\item If a machine has been tripped, the Y matrix is adjusted and reduced every time step. This repeated action could be made more efficient.
	\item Odd IVM behavior may have to do with how PST doesn't really use the synchronous reference frame (\verb|g.sys.syn_ref|) or assumes \verb|g.sys.sys_freq| is always 1.
\end{enumerate}
