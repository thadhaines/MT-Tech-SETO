% file to contain information on loose ends or things generaly unfinished
%==============================================================================
\chapter{Loose Ends - WIP} \ \\
As software development is never actually over, this chapter is meant to contain any loose ends that felt relevant.
The below is copied from the most recent weekly (09/02/20) and some minor additions

\begin{enumerate}
	\item As infinite buses don't seem to be used in dynamic simulation, they were not converted to use the golbal g.
	\item \verb|tgh| model not converted for use with global g. (no examples of tgh gov)
	\item In original (and current) \verb|s_simu|, the global \verb|tap| value associated with HVDC is over-written with  a value used to compute line current multiple times. \\It probably shouldn't be.
	\item Constant Power or Current loads seem to require a portion of constant Impedance.
	\item PSS design functionality not explored
	\item No examples of of delta P omega filter or user defined damping controls for SVC and TCSC models
	\item Differences in \verb|mac_ind| between pst 2 and 3 seem backward compatible - untested.
	\item DC is not implemented in VTS - Just combine into main routine? Seems counter intuitive to do multi-rate variable time step integration.
	\item AGC capacity should consider defined machine limits instead of assuming 1 PU max.
	\item AGC should allow for a `center of inertia' frequency option instead of the weighted average frequency.
	\item A method to initialize the power system with tripped generators should be devised and occur before the first power flow solution.
	\item A method to zero derivatives of any model attached to a tripped generator should be created to enable VTS to optimize time steps.
	\item Re-initializing a tripped generator in VTS will likely require indexing the \verb|g.vts.stVec|. This could be aided by adding indices to the \verb|g.vts.fsdn| cell.
	\item miniWECC DC lines (modeled as power injection) is not included in AGC calculations as the power does not travel over simulated lines.
\end{enumerate}
