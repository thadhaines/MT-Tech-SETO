% file to contain information on the history of PST versions
%==============================================================================
\chapter{PST Version History}

%If it seems wise to include a history about previous PST versions, this file can be included.

% copied from versioning document 09/03/20, modified as deemed fit

%\section*{Current Versions}
\hspace{-4.2em}
\begin{minipage}{.666\linewidth}
\begin{itemize}
\raggedright
\itemsep 0 em
\singlespacing
\item 1.0 - Original PST circa 1991.
\item 2.0 - Updated PST circa 1999.
\item 2.1 - Dr. Dan Trudnowski added power and current injection via the pwrmod model circa 2015.
\item 2.2 - Dr. Dan Trudnowski added arbitrary generator tripping functionality circa 2019.
\item 2.3 - Dr. Dan Trudnowski added a voltage behind a reactance model as ivmmod circa 2019.
\item 3.0 - Updated PST obtained from Joe Chow's website circa 2020. 
Includes fixes, alterations, and MIT license.
Addition of multiple DC lines and PSS modifications.
\item 3.1 - Based on 3.0, incorporates Dr. Trudnowski's pwrmod and ivmmod models in non-linear simulation, machine tripping functionality, and various model patches.
pwrmod included in linear simulation as well. 
\item SETO - Based on 3.1. Experimental version by Thad Haines using a new global structure, automatic generation control, and variable time step (VTS) integration. 
\item 3.1.1 - Based on 3.1. Version by Ryan Elliott at Sandia National Labs with energy storage and updated linear simulation along with various other fixes and code cleanup alterations. 
\item 3.1.2 - Based on 3.1.1. Work in progress version by Ryan Elliott to clean up global variables.

%\item 4.0.0-aX - Alpha version of PST 4 based on SETO version. 
%Includes a refined VTS routine, confirmed multi-generator tripping, and improved AGC action/modulation.

%Examples in repository will be re-worked/updated and cleaned to use this version, 
%documentation of changes will be updated, 
%and PST code will be cleaned up where possible.
%May go into beta, which may then go into the `release candidate' phase, but may also just turn into 4.0.0.
%\end{itemize}

%\section*{Possible Future Versions}
%\begin{itemize}
\itemsep 0 em
\item 4.0.0 - Released circa October 2020. Based on SETO version. 
Includes a refined VTS routine, confirmed multi-generator tripping, improved AGC action and modulation, code cleanup, example library, and documentation.
Represents the end result of four months of work by Thad Haines.
%\item 4.1.0 - Based on 4.0.x - planned to incorporate energy storage models for non-linear simulation and possibly the updated linear simulation based on 3.1.1. Examples of \verb|ess| model use required.
\end{itemize}
\end{minipage}%
\begin{minipage}{.42\linewidth}
\begin{center}
{\singlespacing 
%===============================================================================
\begin{tikzpicture}[node distance=2cm] 
%original versions
\node[shape=circle, draw, inner sep=2pt] (1) {1.0};
\node[shape=circle, draw, inner sep=2pt, below of=1] (2) {2.0};
% trudnowski versions
\node[shape=circle, draw, inner sep=2pt, below right of=2] (21) {2.1};
\node[shape=circle, draw, inner sep=2pt, below of=21] (22) {2.2};
\node[shape=circle, draw, inner sep=2pt, below of=22] (23) {2.3};
%updated chow version
\node[shape=circle, draw, inner sep=2pt, below left of=23] (3) {3.0};
% cobmination of 2.3 and 3
\node[shape=circle, draw, inner sep=2pt, below right of=3] (31) {3.1};
% start of thad versions
\node[shape=circle, draw, inner sep=2pt, below of=31] (seto) {SETO};
\node[shape=circle, draw, inner sep=2pt, below of=seto] (4) {4.0.0};
% ryan versions
\node[shape=circle, draw, inner sep=2pt, below right of=31] (311) {3.1.1};
\node[shape=circle, draw, inner sep=2pt, below of=311] (312) {3.1.2};

%---- connecting lines
% OG
\draw [arrow] (1) -- (2);
\draw [arrow] (2) -- (3);
% dan
\draw [arrow] (2) -- (21);
\draw [arrow] (21) -- (22);
\draw [arrow] (22) -- (23);
% thad
\draw [arrow] (3) -- (31);
\draw [arrow] (23) -- (31);
\draw [arrow] (31) -- (seto);
\draw [arrow] (seto) -- (4);
% ryan
\draw [arrow] (31) -- (311);
\draw [arrow] (311) -- (312);

%===============================================================================
\end{tikzpicture}
}
\begin{figure}[H]
	\centering
	\footnotesize
	\caption{Development of PST.}
	\label{fig: pst history}
\end{figure}%\vspace{-1 em}
\end{center}
\end{minipage}
