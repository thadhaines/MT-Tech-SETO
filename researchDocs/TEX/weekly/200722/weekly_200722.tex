\documentclass[12pt]{article}
\usepackage[english]{babel}
\usepackage[utf8]{inputenc}

%% Pointer to 'default' preamble
% pacakages and definitions

\usepackage{geometry}
\geometry{
	letterpaper, 
	portrait, 
	top=.75in,
	left=.8in,
	right=.75in,
	bottom=.5in		} 	% Page Margins
	
%% additional packages for nice things
\usepackage{amsmath} 	% for most math
\usepackage{commath} 	% for abs
\usepackage{lastpage}	% for page count
\usepackage{amssymb} 	% for therefore
\usepackage{graphicx} 	% for image handling
\usepackage{wrapfig} 	% wrap figures
\usepackage[none]{hyphenat} % for no hyphenations
\usepackage{array} 		% for >{} column characterisctis
\usepackage{physics} 	% for easier derivative \dv....
\usepackage{tikz} 		% for graphic@!
\usepackage{circuitikz} % for circuits!
\usetikzlibrary{arrows.meta} % for loads
\usepackage[thicklines]{cancel}	% for cancels
\usepackage{xcolor}		% for color cancels
\usepackage[per-mode=fraction]{siunitx} % for si units and num
\sisetup{group-separator = {,}, group-minimum-digits = 3} % additional si unit table functionality

\usepackage{fancyhdr} 	% for header
\usepackage{comment}	% for ability to comment out large sections
\usepackage{multicol}	% for multiple columns using multicols
\usepackage[framed,numbered]{matlab-prettifier} % matlab sytle listing
\usepackage{marvosym} 	% for boltsymbol lightning
\usepackage{pdflscape} 	% for various landscape pages in portrait docs.
%\usepackage{float}
\usepackage{fancyvrb}	% for Verbatim (a tab respecting verbatim)
\usepackage{enumitem}	% for [resume] functionality of enumerate
\usepackage{spreadtab} 	% for using formulas in tables}
\usepackage{numprint}	% for number format in spread tab
\usepackage{subcaption} % for subfigures with captions
\usepackage[normalem]{ulem} % for strike through sout

% for row colors in tables....
\usepackage{color, colortbl}
\definecolor{G1}{gray}{0.9}
\definecolor{G2}{rgb}{1,0.88,1}%{gray}{0.6}
\definecolor{G3}{rgb}{0.88,1,1}

% For table formatting
\usepackage{booktabs}
\renewcommand{\arraystretch}{1.2}
\usepackage{floatrow}
\floatsetup[table]{capposition=top} % put table captions on top of tables

% Caption formating footnotesize ~ 10 pt in a 12 pt document
\usepackage[font={small}]{caption}

%% package config 
\sisetup{output-exponent-marker=\ensuremath{\mathrm{E}}} % for engineer E
\renewcommand{\CancelColor}{\color{red}}	% for color cancels
\lstset{aboveskip=2pt,belowskip=2pt} % for more compact table
%\arraycolsep=1.4pt\def
\setlength{\parindent}{0cm} % Remove indentation from paragraphs
\setlength{\columnsep}{0.5cm}
\lstset{
	style      = Matlab-editor,
	basicstyle = \ttfamily\footnotesize, % if you want to use Courier - not really used?
}
\renewcommand*{\pd}[3][]{\ensuremath{\dfrac{\partial^{#1} #2}{\partial #3}}} % for larger pd fracs
\renewcommand{\real}[1]{\mathbb{R}\left\{ #1 \right\}}	% for REAL symbol
\newcommand{\imag}[1]{\mathbb{I}\left\{ #1 \right\}}	% for IMAG symbol
\definecolor{m}{rgb}{1,0,1}	% for MATLAB matching magenta
	
%% custom macros
\newcommand\numberthis{\addtocounter{equation}{1}\tag{\theequation}} % for simple \numberthis command

\newcommand{\equal}{=} % so circuitikz can have an = in the labels
\newcolumntype{L}[1]{>{\raggedright\let\newline\\\arraybackslash\hspace{0pt}}m{#1}}
\newcolumntype{C}[1]{>{\centering\let\newline\\\arraybackslash\hspace{0pt}}m{#1}}
\newcolumntype{R}[1]{>{\raggedleft\let\newline\\\arraybackslash\hspace{0pt}}m{#1}}

%% Header
\pagestyle{fancy} % for header stuffs
\fancyhf{}
% spacing
\headheight 29 pt
\headsep 6 pt

%% Header
\rhead{Thad Haines \\ Page \thepage\ of \pageref{LastPage}}
\chead{Research Update \\ Week of July 22nd, 2020}
\lhead{Research \\ }

\usepackage[hidelinks]{hyperref} % allow links in pdf
%\href{https://digitalcommons.mtech.edu/grad_rsch/242/}{Long-Term Dynamic Simulation of Power Systems using Python, \\Agent Based Modeling, and Time-Sequenced Power Flows}
\begin{document}
\begin{multicols}{2}
\raggedright

\paragraph{Recent Progress:}
	\begin{enumerate}
		\itemsep0em 
		\item Global g,non-linear functionality added for: %  linear \& 
		\begin{minipage}{\linewidth}
		\begin{multicols}{2}
				\begin{itemize}
		\itemsep0em 
				\footnotesize
				\raggedright
					\item interface values (Y matricies)
					\item line
					\item bus
					\item areas
					\item line monitor (lmon)
					\item AGC
				\end{itemize}
		\end{multicols}
		\end{minipage}
		\item Fixed limiting issue in lmod/rlmod.
		\item Created line monitor, area and AGC models/functionality.
		\item \href{https://github.com/thadhaines/MT-Tech-SETO/tree/master/PST/0-examples/AGC}
		{Created AGC example}
		\item \href{https://github.com/thadhaines/MT-Tech-SETO/blob/master/researchDocs/TEX/one-offs/200720-PSTandAGC/200720-PSTandAGC.pdf}
		{Lightly documented AGC example}
		\item \href{https://github.com/thadhaines/MT-Tech-SETO/blob/master/researchDocs/TEX/one-offs/200709-PSTsetoVersionChanges/200709-PSTsetoVersionChanges.pdf}
		{Updated pst SETO change doc}
		
		%\item Added Exciter models 0-4 to global g, tested as functional in both linear and non-linear batch runs.
		%\item Started work on batch unit testing
		%\item ODE test for variable time step\
		%\item Found example for SVC - shows version difference between 2 and 3 - possibly due to \verb|exc_dc12| model.
		%\item Created more MATLAB plot functions to compare PST data
		\item \href{https://github.com/thadhaines/MT-Tech-SETO}{GitHub updated:}\\
	{\footnotesize \verb|https://github.com/thadhaines/MT-Tech-SETO| }\\
	\end{enumerate}
	

\paragraph{Sandia Action Items:}
	\begin{itemize}
		\itemsep 0em 
			\item Continue development of pwrmod / ivmmod models and their implementation in PST.
			%\item Be aware of multi rate methods % mostly TAMU/Power world
			\item Decide on PST base version (3.1$\longrightarrow$SETO)
			\item Plan for variable time step methods
			\item Investigate power electronics-based models (REGC - Matt)
	\end{itemize}
	
\paragraph{Current Questions:}
	\begin{enumerate}
	\itemsep0em 
	
		\item Differences in \verb|mac_ind| between versions.
	\item Induction machines have no speed?\\ only angle?
	\item PST modeling of transformers?
	\item Play in data for variable solar irradiance?
	\item PSS design doesn't seem to be used in normal simulation?
	\item Deadlines of any sort?
	\item Continued employment beyond \\August 12th?
	\end{enumerate}

\vfill\null
\columnbreak

	
\paragraph{Current Tasks:}
	\begin{enumerate}
		\itemsep 0em 
		\item Think about using standard ODE solvers
				\item Requirements for variable step methods:
						\begin{minipage}{\linewidth}
							\begin{itemize}
								\itemsep0em 
								\footnotesize
								\item Functionalized Network solution
								\item Functionalized Derivative calculation
								\item Functionalized collection and return of calculated derivatives
								\item `outputfunction' that handles logging of correct output values and indexing
								\item updated scheduler to run simulation in `blocks' between known events.
							\end{itemize}
						\end{minipage}
		\item Decisions concerning remaining globals:
		\begin{minipage}{\linewidth}
				%\begin{multicols}{2}
						\begin{itemize}
				\itemsep0em 
						\footnotesize
							\item IVM (waiting for linear code)
							\item delta P omega filter (no examples)
							\item PWR (cell data only)
							\item pss design (not used in simulation)
						\end{itemize}
				%\end{multicols}
				\end{minipage}
		\item Refine AGC implementation.
		\begin{minipage}{\linewidth}
				%\begin{multicols}{2}
						\begin{itemize}
				\itemsep0em 
						\footnotesize
				\item update/create pstSETO flowchart
				\item create algorithm flowchart
				\item Add conditional ACE
				\item Account for tripped gens H removal.
							
						\end{itemize}
				%\end{multicols}
				\end{minipage}
		\item Think about cleaning up or flowcharting svm\_mgen\_Batch
		%\item Reference Trudnowski code for `structured array, functionalized' approach.
		%\item Reference Stajcar code for basic transient stability simulation flow.
		\item Work on understanding PST operation
		\item Document findings of PST functionality
		\item Investigate Octave compatibility
\end{enumerate}



\paragraph{Coding Thoughts:} 
	\begin{enumerate}

		\itemsep 0em 
		\item Condense $\approx$340 globals into 1 structured array with $\approx$18 fields based on category.
		\item Create new \verb|s_simu_Batch| style script that functionalizes the newtork and dynamic calculations so that standard MATLAB ODE solvers may be used.
		%\item Enable `objects' (structure of arrays), but include functions to interact with condensed globals so vectorized operations are still possible.\\
	%	This requires more conceptual modeling to understand what needs to be passed/references/changed for each `object'.
	%	Would enable addition of area definitions to models.
	%	\item Separate total system calculation of derivatives into scripts/functions to allow for easier changing of integration method.
	%	Possibly employ \verb|feval| for a more dynamic calculation routine.
		\item Rework how switching \& perturbance events are handled into a more flexible and general format. (flags? objects?)
		\item Generate something similar to unit test cases to verify code changes don't break everything during refactor.
		\item Generate comparison scripts to verify simulated results match after code changes.

	\end{enumerate}
	





\vfill\null
\end{multicols}



\begin{comment}
\paragraph{Possible Future Tasks:} % Maybe good ideas?
	\begin{enumerate}
		\item Investigate Sandia integrator stability methods.	
		See if the modified PST used by Sandia in 2015 paper exists for an example of how they implemented different integration routines / stability calculations.	
	\end{enumerate}



		\item PST modeling of faults: \\
		Uses alternate Y matrices? \\
		Creates fault bus?

Things required for simulation....
\begin{minipage}{\linewidth}
\begin{multicols}{2}
		\begin{itemize}
\itemsep0em 
		\footnotesize
			\item system model
			\item Load flow solver
			\item Network solver
			\item Machine models
			\item Governor model
			\item Exciter models
			\item converter models
			\item load modulation models
			\item ...
		\end{itemize}
\end{multicols}
\end{minipage}

% Unused categories
\paragraph{Future Tasks:} %(Little to No Progress since last time / Things coming down the pipe)
	\begin{enumerate}
		\item none		
	\end{enumerate}
	
\paragraph{Future Work: (not by me)}
	\begin{itemize}
		\item none
	\end{itemize}

\paragraph{Requests:}
	\begin{enumerate}
			\item none
	\end{enumerate}
\end{comment}



\end{document}